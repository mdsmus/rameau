\documentclass{article}
\usepackage{graphicx}
\usepackage{url}
\usepackage[utf8x]{inputenc}
\usepackage[T1]{fontenc}
\usepackage[english]{babel}
\usepackage{color}
\usepackage{times}

\title{Rameau Programmer's Guide}
\author{Pedro Kroger and Alexandre Passos}

\newcommand{\function}[2]{
  \noindent\texttt{#1}\hfill\textbf{[function]}\\
  #2
\vspace{2em}
}

\newcommand{\example}[2]{
  \par Example: \texttt{#1} $\rightarrow$ #2
}

\begin{document}
\maketitle

\section{Musiclib}
\label{sec:musiclib}

Musiclib is a library to deal with music information and operations.

\subsection{Assessors functions}
\label{sec:accessors-functions}

The following functions foo bar ....

;; note-code is a list representing a note name and it's accidentals.
;;   Examples: (c 2)  is C## (c double-sharp)
;;             (d -1) is Db (d flat)
;; interval-code is alist representing an interval


\section{Events}
\label{sec:events}

;; Formato interno:
;; O formato interno é uma lista de eventos. Cada evento é uma nota
;; que soa, e soa em uma altura (pitch), por um certo tempo (dur) a
;; partir de um certo instante na música (inicio). Isso é representado
;; na struct evento.

;; Antes disso, no entanto, as notas são processadas pra se extrair a
;; duração e a altura. Isso é feito usando a struct nota, que
;; desaparece depois do primeiro passo de processamento, quando as
;; notas são transformadas em eventos.

;; As funções desse arquivo só fazem essa conversão, de notas pra
;; eventos, e representam esses eventos, e os manipulam de forma
;; básica.


\input{musiclib}

\section{Format}
\label{sec:format}

;; The atomic structures returned by the parser are \\texttt{event}s,
;; \\texttt{note-sequence}s and \\texttt{note-simultaneous}. These are
;; distinct from their AST counterparts defined in parser.lisp.
;;


\end{document}
