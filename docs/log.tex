\documentclass[12pt,brazil]{book}
\usepackage[utf8x]{inputenc}
\usepackage{babel}
\usepackage[top=2.5cm,left=2.5cm,bottom=2.5cm,right=2.5cm]{geometry}
\usepackage{url}
\usepackage{graphicx}
\usepackage{html}

\title{Rameau: logfile}
\author{Pedro Kröger}

\begin{document}
\graphicspath{{figs/}}

\maketitle

\chapter{Histórico}
\label{cha:historico}

\section{Visão geral}
\label{sec:visao-geral}

\begin{tabular}{ll}
  13 março 2007  & início do repositório git \\
                 & trabalho no parser básico de lilypond \\
  17 julho 2007  & início da implementação de pardo \\
  09 agosto 2007 & algoritmo de pardo gera gabaritos \\
                 & refatoração do código e melhorias no parser \\
  21 agosto 2007 & corais de bach e kostka-payne no repositório \\
  26 agosto 2007 & sistema trac \\
  28 agosto 2007 & inicio dos gabaritos \\
  01 setembro 2007 & início musiclib \\
  16 setembro 2007 & versão ``final'' de musiclib \\
                   & início de refatoração \\
  25 setembro 2007 & rameau usa asdf (finalmente!) \\
  27 setembro 2007 & primeira implementação do formato pop \\
  29 setembro 2007 & início de modo do emacs para formato pop \\
  30 setembro 2007 & programa rameau-testes agora é binário \\
                   & mais refatoração e correção de bugs \\
  05 outubro 2007  & início do porting do código de temperley \\
                   & otimizações no código e profiling \\
  08 outubro 2007  & suporte para cmucl e clisp \\
  09 outubro 2007  & melhorias no parser \\
  13 outubro 2007  & fim dos bugs com pacotes e simbolos \\
  14 outubro 2007  & rameau 1.0 \\
  06 novembro 2007 & rameau 1.1 \\
  11 novembro 2007 & rameau 1.2 \\
\end{tabular}

\section{Gabarito}
\label{sec:gabarito}

\section{Análise}
\label{sec:analise}

a escrita dos corais de bach é contrapontistica, então é fácil confundir
as coisas harmonicamente. o exemplo anexo é tirado do coral 17 da edição
do riemenschneider. o coral é em sol maior e o exemplo é o começo da 4a
frase (a frase anterior terminava co uma cap em sol).

\includegraphics{exemplo}

nesse trecho o mais fácil (e logico) é pensar nas colcheias si no 2o
tempo e c$\sharp$ no terceiro tempo como notas de passagem, desse modo temos:

V viio V6 I

por outro lado, se consideramos o si colheia do segundo tempo, o
acorde seria Em. Outra possibilidade seria C$\sharp$ø7/E "sem a
tônica" (mas ela foi ouvida logo antes, e esse é um idioma comum).
Isso vale para o acorde na colcheia em c$\sharp$ no 3o tempo. Ele
seria um F$\sharp$m, mas outros podem ouvir como um D7M (mesmo caso
"sem tonica").

o que voce acha melhor considerar:

1. notas de passagem

2. acorde de setima sem a tonica, considerando que a tonica foi ouvida
   logo antes

3. acordes formados naquele momento (i.e. Em e F$\sharp$m)

estamos tentando ser completos no gabarito, então se o computador
escolher 1 ou 2 estaria correto. o que voce acha?

\chapter{Corais}
\label{chap:corais}

\section{Objetivo}
\label{sec:objetivo}

O objetivo principal é ter os 371 corais harmonizados de Bach no
formato do lilypond de modo que possa servir para diversas pesquisas.
Esses arquivos devem ser revisados e corrigidos. Os arquivos devem ser
lançados sob a licensa GPL.

\section{Introdução}
\label{sec:introducao}

%% falar mais da importancia dos corais, citar McHose
Os corais harmonizados de J. S. Bach são utilizados frequentemente
para o estudo da harmonia ocidental tradicional. Os corais são
excelentes para o estudo da harmonia já que apresentam um número
grande de mudanças harmônicas em pouco tempo, enquanto uma sinfonia
pode demorar vários compassos em apenas 2 ou 3 acordes.

Não existe um repositório púpblico dos 371 corais harmonizados em
formato simbólico. Donald Byrd cita os corais de Bach na sua lista de
coleções úteis para \textit{Music Information Retrievel}
(\url{www.informatics.indiana.edu/donbyrd/MusicTestCollections.HTML}).
A coleção em
\url{kern.humdrum.org/cgi-bin/kern/kssearch?s=t&keyword=Bach+Johann}
possui alguns corais no formato Kern usado pelo Humdrum. A coleção em
\url{www.musedata.org/encodings/bach/bg/chorals/} tem 185 corais no
formato musedata. Até onde sabemos os arquivos no musedata tem
limitações enquanto ao uso. Finalmente, \url{www.jsbchorales.net}
contém todos os 371 corais no formato MIDI. Esse site também contém
várias meta-informações úteis como os títulos dos corais, número de
BWM, número em difentes edições, etc.

Nós decidimos seguir a ordem dos corais usada na edição de Albert
Riemenschneider. Os corais em formato MIDI foram baixados de
\url{www.jsbchorales.net} e convertidos para o formato do lilypond com
o script midi2ly, distribuído com o lilypond. Um script foi criado
para usar a informação em \url{www.jsbchorales.net/riemen.shtml} e
re-nomear os arquivos para o seu numero correspondente na edição de
Riemenschneider. Por exemplo, o arquivo \texttt{026900b\_.mid} foi
renomeado para 001.mid.

A conversão de midi para lilypond funcionou perfeitamente, contudo
midi2ly gera um arquivo seguindo a estrutura do arquivo midi. Nesse
caso os arquivos gerados foram consideravelmente prolixos e difíceis
de ler, ainda que nosso parser conseguisse ler e analisa-los
corretamente. Um trecho do coral 002 pode ser visto abaixo. Ele foi
editado para economizar espaço, e apenas a primeira linha das notas
foi mostrada.

\begin{verbatim}
trackAchannelA =  {
  \time 4/4 
  \key a \major
  \tempo 4 = 76 
}

trackA = <<
  \context Voice = channelA \trackAchannelA
>>

trackBchannelA = \relative c {
  % [SEQUENCE_TRACK_NAME] Instrument 1
  s2. a''4 |
  .... (notas)
}

trackB = <<
  \context Voice = channelA \trackBchannelA
>>

trackCchannelA =  {
  % [SEQUENCE_TRACK_NAME] Instrument 2
}

trackCchannelB = \relative c {
  s2. e'4 |
  .... (notas)
}

trackC = <<
  \context Voice = channelA \trackCchannelA
  \context Voice = channelB \trackCchannelB
>>

trackDchannelA =  {
  % [SEQUENCE_TRACK_NAME] Instrument 3
}

trackDchannelB = \relative c {
  s2. cis'4 |
  .... (notas)
}

trackD = <<
  \clef bass
  \context Voice = channelA \trackDchannelA
  \context Voice = channelB \trackDchannelB
>>


trackEchannelA =  {
  % [SEQUENCE_TRACK_NAME] Instrument 4
}

trackEchannelB = \relative c {
  s2. a'8 gis |
  .... (notas)
}

trackE = <<
  \clef bass
  \context Voice = channelA \trackEchannelA
  \context Voice = channelB \trackEchannelB
>>

\score {
  <<
    \context Staff=trackB \trackB
    \context Staff=trackC \trackC
    \context Staff=trackD \trackD
    \context Staff=trackE \trackE
  >>
}
\end{verbatim}

Para facilitar a leitura, edições, e correções posteriores por
humanos, desenvolvemos um script em lisp para converter os arquivos no
formato acima para algo mais legível. O resultado pode ser visto
abaixo:

\begin{verbatim}
\header {
  title = "2 - Ich dank dir, lieber Herre"
  composer = "J. S. Bach"
}

global =  {
  \time 4/4 
  \key a \major
}

soprano = \relative c {
  \partial 4 a''4 
  a a a b 
 ... notas ...
}

alto = \relative c {
  \partial 4 e'4 
  fis e fis fis 
 ... notas ...
}

tenor = \relative c {
  \partial 4 cis'4 
  cis cis8 b a gis fis4 
 ... notas ...
}

baixo = \relative c {
  \partial 4 a'8 gis 
  fis4 cis d dis 
 ... notas ...
}

\score {
  <<
    \new Staff {
      <<
        \global
        \new Voice = "1" { \voiceOne \soprano }
        \new Voice = "2" { \voiceTwo \alto }
      >>
    }
    \new Staff {
      <<
        \global
        \clef "bass"
        \new Voice = "1" {\voiceOne \tenor }
        \new Voice = "2" { \voiceTwo \baixo \bar "|."}
      >>
    }
  >>
}
\end{verbatim}

Observe que cada voz é claramente marcada com nomes legíveis como
``soprano'' ou ``alto''. O script usa informação da lista on-line para
inserir o título do coral. Finalmente, os corais são montados em dois
pentagramas, como na edição do Riemenschneider.

O script também lidou com anacruses. O arquivo original tinha uma
pausa escondida (um \textit{skip}) para a anacruse. O script
interpretou essa informação e colocou o comando \texttt{partial} para
fazer anacruzes da maneira correta.

Um dos maiores problemas com o MIDI é que informações enarmônicas são
perdidas. O script procurou fazer as enarmonizações mais obvias,
baseando-se na indicação de armadura. Por exemplo, uma nota que foi
escrita como Sol$\sharp$ na tonalidade de Mi$\flat$ maior foi
corretamente enarmonizada para Lá$\flat$. Contudo notas fora da
armadura não foram enarmonizadas automaticamente e deverão ser feitas
manualmente.

Surpreendentente alguns corais tem mudança de compasso implícita, i.e.
sem indicação de nova fórmula de compasso (e.g. coral 359). Decidimos
indicar todas as mudanças de tempo explicitamente.

A edição de Riemenschneider tem notas extras em alguns corais,
indicando outras possibilidades de execução (especialmente na voz do
baixo) ou para indicar um instrumento. Nos escolhemos retirar essas
vozes para concentrar apenas na textura a 4 vozes.

Após essa etapa os corais ainda tem problemas. 

Na edição impressa alguns corais tem indicação de repetição. Nos
arquivos MIDI as repetições são feitas por extenso. Isso faz com que
os arquivos lilypond fiquem maiores do que necessário e a análise
harmônica tenha que ser feita duas vezes para um mesmo trecho. Além
disso, a impressão do coral em pdf fica desnecessariamente grande.

O programa midi2ly converteu algumas notas longas na sintaxe
\texttt{c8*4} que indica que uma nota dura 4 vezes o valor marcado
(nesse caso 4 colcheias, ou 1 mínima). Na partitura ela aparece como
uma colcheia mas ocupa o lugar de uma mínima. É necessário converter
essa notação para algo simplificado como c2 o c4~c4 se a nota não
couber no compasso.

Para resolver esses problemas fizemos outro script em lisp. Ele lê os
arquivos já convertidos pelo primeiro script e guarda informações em
uma estrutura. Nesse script foram criadas funções para procurar e
eliminar as repetições nos corais. O script procura simultaneamente
nas 4 vozes, para garantir que a repetição será a mesma nas 4 vozes.
Para garantir acuracidade, uma lista de todos os corais que não tem
repetição foi feita manualmente (uma tarefa entediante mas mais rápida
que parece; levou 20 minutos). Uma função retornava uma lista com os
corais com repetição e outra sem. Desse modo foi possível comparar com
a lista feita manualmente e encontrar problemas no programa ou nos
arquivos com os corais. Durante esse processo foram encontradas notas
erradas em alguns corais. O código para buscar repetições é bastante
simples e pode ser visto abaixo. A função \texttt{voz-pos} retorna a
posição onde as 4 vozes repetem. Essa possição é usada por
\texttt{voz-subseq} para retorna uma lista sem a repetição de cada
voz.

\begin{verbatim}
(defun voz-subseq (voz pos)
  (subseq voz (if pos pos 0)))
  
(defun search-pos (list x)
  (let ((list (mapcar #'parse-nota list)))
    (search (subseq list 0 x) list :key #'first :test #'equal :from-end t)))

(defun voz-pos (l1 l2 l3 l4)
  (loop
     for x from 19 downto 10
     for a1 = (search-pos l1 x)
     for a2 = (search-pos l2 x)
     for a3 = (search-pos l3 x)
     for a4 = (search-pos l4 x) do
       ;;(print (list x a1 a2 a3 a4))
       (when (and (plusp a1) (plusp a2) (plusp a3) (plusp a4))
         (return a1))))
\end{verbatim}

Os corais 043, 055, 087, 121, 205, 253, 327, 333, 272, 218, 222, e 323
tem repetições mais complexas como casa 1 e 2, mais de uma repetição,
etc. Eles deverão ser editados manualmente.

\section{Problemas}
\label{sec:problemas}

O coral 248 não corresponde ao coral no riamanscheiner. 5 dos 371
corais não puderam ser corrigidos automaticamente; os corais 308, 236,
e 215 estão fora do formato acima e não puderam ser convertidos. O
coral 48 estava tão mal-codificado que terá que ser feito manualmente.
O coral 150 tem 1 voz faltando (ele é a 5 vozes).


\end{document}

